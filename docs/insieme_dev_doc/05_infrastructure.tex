\chapter{The Infrastructure} \label{cap:infrastructure}

\section{The Insieme Wiki}
\section{The Source Code Repository}
\section{Third-Party Libraries and Install Scripts}

\section{The Build Script} \label{sec:Infrastructure.Build}

\section{The Integration Tests [Simone]}

Integration tests are used to test the Insieme compiler as a single entity,
checking that each compiler phase is working properly and correctly
interoperates with the others . The underlying idea is to
use real C/C++ codes, convert them into semantically correct IR and through the
backends produce C code which when executed produces the same output as the
original input code. At each step we verify that the intermediate results are
consistent with reference files which have been generated in the previous {\em
correct} run of the test. 

[when the master student assigned for this task will deliver his thesis, it can
be integrated here]

\subsection{Folder Organization}
Test cases are organized in a hierarchal directory structure. The leaf
directories contains the actual test cases whereas higher order directory are
used as logic containers with the goal to group test cases belonging to the same
category, e.g. the non-leaf {\tt ocl} directory contains OpenCL test cases. The
location of the integration test cases within the Insieme compiler project is in
the \file{test} folder. 

The integration test script recursively scans directories collecting all test
cases which are the leaves of the directory structure. However this behaviour
can be overloaded by adding a file called \file{test.cfg} into any
non-leaf directory. When this file is present the test script avoids to scan all
the sub folders from the current location, it instead reads the file content and
recurs only on the folder names listed in the \file{test.cfg} file.
The structure of the \file{test.cfg} must be a new-line separated list of
directory names which are direct sub folders of the current directory. See
\file{test/test.cfg} for an example. A line within the \file{test.cfg} file can
be commented by using the {\tt \#} symbol. 

\subsection{Test case Organization and Configuration (leaf directories)}
As stated, each leaf folder contains a test case. There are several ways for
declaring a new test case varying from basic to more advanced configurations. In
advanced configurations files are included within the test case folder which
will be interpret by the test script. 

The minimum test case consists of a single file ({\tt .c} or {\tt .cpp}) having
exactly the same name of the enclosing test case folder. An example is the
\file{test/hello\_world} test case example. The directory contains a single
file, i.e. \file{hello\_world.c}.

The default behaviour of the test script can be overloaded by means of files
whose semantics is described hereafter. Note that the name must be the same as
described here (case sensitive).

\begin{description}

	\item[\file{inputs.data}:]
		However, sometimes a test case is composed by multiple translation units. In
		order to allow such codes the user can define a file called \file{inputs.data}
		within the test case folder. This file must contain a space separated list of
		input source files ({\tt .c} or {\tt .cpp}). When this file is present, the
		integration test script will not look for a file having the same name of the
		current folder but instead it will only consider the ones provided in this file. 

	\item[\file{insieme.flags}:] For a test case we may want to enable a specific
		set of flags of the insieme compiler or provide a pre-processor directive
		required during the compilation process. The file \file{insieme.flags}
		allows the programmer to provide custom settings in a per-test case way. 

	\item[\file{ref-gcc.flags}:] Sometimes beside providing specific flags to the
		Insieme compiler, we need to customize the way the original code is
		compiled. The file \file{ref-gcc.flags} allows the programmer to specify
		such flags. It is worth notice that the flags specified here are only used
		to compile the original code and not the code which is generated as output
		of the Insieme source-to-source process. 

	\item[\file{test-gcc.flags}:] Compiler configurations which are applied
		to code generated by the Insieme backend are instead provided in the
		\file{test-gcc.fags} file. An example of test case using all of these
		features is the \file{test/ocl/nbody} test case. 

	\item[\file{prog.input}:] By default the input program is compiled into an
		executable and then executed. However, in some cases the executable needs
		input arguments to be passed or it has to be executed through a
		third-party script (e.g. {\tt mpirun}). When the file \file{prog.input} is
		present, the command line string to execute the command is read from that
		file. Within the file the programmer can use {\tt \{EXEC\}} to refer to
		the executable (since different executable are often generated depending
		on the selected backend). Additionally the {\tt \{PATH\}} placeholder can
		be use to refer to the current path. An example of the content of this
		file is the following: 
		\begin{verbatim} 
		mpirun -n 2 {PATH}/{EXEC} arguments
		\end{verbatim} 

\end{description}

\subsubsection{Passes}
The last file name which is interpreted by the integration test case is the
\file{passes} file. The integration test script defines several passes which can
be applied to a test case, default passes are the following:

\begin{description}

	\item[\texttt{gcc}:] Runs GCC on the test case source code, generate an executable and
		runs it.

	\item[\texttt{sema}:] Runs the Insieme compiler on the test case source code,
		generates the corresponding IR and perform the semantic checks.

	\item[\texttt{c\_seq}:] Runs the Insieme compiler on the test case source code,
		generates the corresponding IR, invokes the sequential backend
		(transforming IR code into C code), compiles and run the generated code.

	\item[\texttt{c\_run}:] Similar to the previous pass, but this time the output C
		code is produced using the runtime backend and executed in parallel. 

	\item[\texttt{ocl}:] Uses the OpenCL backend and runtime. 

\end{description}

We will show later how new passes can be easily defined to take into account new
scenarios (e.g. C++ integration tests, MPI integration tests). Clearly not all
passes can be used with every test case, e.g. it makes little sense to run a
sequential test case using the OpenCL backend and runtime system. The
\file{passes} file used in order to let the integration test script know which
{\em passes} should be used for a particular folder (and sub-folders).

The structure is again fairly easy, the file contains the list of passes, using their
names, which should be enabled. It is also possible to disable a specific pass
(probably enabled at a higher level) prepending the {\tt \char`\^} symbol to the pass
name. For example the string:
\begin{verbatim} 
gcc sema ^c_run
\end{verbatim} 
Enables the {\tt gcc} and the {\tt sema} pass and disable the {\tt c\_run}. When
the integration test will execute the test cases contained within this folder
only the first two configurations will be enabled the runtime backend will not
be tested (unless overwritten in one of the sub-folders).

Passes files can be place either in test case folders (leaf-directories) or in
higher levels. Specifications at the lower level always precedes the more
general configuration found on parent folders.  

\subsection{Running the Integration tests} 

After configured the environment with {\tt cmake} a \file{integration\_tests.py}
file will be available in the build folder. Launching the script can be done
with:

\begin{verbatim}
$> ./integration_tests.py 
\end{verbatim}

This script is written in Python 3.x, therefore makes sure you have a version of
the Python interpreter which is at least 3.0.  The available options are
accessed through the {\tt --help} command line parameter, the most important
flags are herein explained:

\begin{description}

	\item [{\tt -w NUM}:] Used to specify the number of parallel executors to be
		used for running the test cases. The script runs test cases in parallel to
		reduce the time required for integration tests. Output of the script is
		kept ordered by printing the output of a specific test case only when all
		its parts has been executed. 

	\item [{\tt -c | --clean}:] Removes all the reference files which have been
		generated in the previous run. 

	\item[{\tt -m | --mock-run}:] Produce as output the list of commands which
		are generated by the integration script to test a particular test case.
		The generated commands always contain absolute paths, therefore they can
		be executed from any location of the current machine. 

	\item[{\tt -r NUM}:] Instead of running each executable only once, it runs it
		multiple times (exactly {\tt NUM}) and in the test case report prints
		information on the average execution time and the standard deviation. 

	\item[{\tt -b passes}:] Allows the programmer to specify the list of passes
		which should be used.  \\
		{\bf\tt NOTE:} Because of an implementation issue, the script will still
		prioritize the configuration given by the \file{passes} files over the one
		provided by the command line even though the contrary should be
		preferable.

\end{description}

The script by default load all the test scripts from the \file{test/} folder.
However the path to test cases can be provided via the command line to execute
only a subset of the tests. For example:

\begin{verbatim}
$> ./integration_tests.py test/omp/ test/ocl
\end{verbatim}

Executes all of the OpenMP and OpenCL test cases. Also wildcards are allowed,
for example we can execute all test cases which are direct sub-folders of the
\file{test} directory starting with the letter 'o' as follows:

\begin{verbatim}
$> ./integration_tests.py test/o*
\end{verbatim}

Or as said before, only execute a subset of the passes and run the tests in
parallel using 4 workers:

\begin{verbatim}
$> ./integration_tests.py -b gcc,sema -w 4
\end{verbatim}

\subsection{Adding a new Integration test case} 

Adding a new test is easy. 
\begin{itemize}
	\item First of all find the correct location where to place
		your test. C++ test cases must be placed under the \file{test/cpp} folder.
		OpenMP tests are inside the \file{test/omp} folder. 

	\item Secondly create a directory with the name of the test case, add the
		source files and any other configuration file required to implement your
		test. 

	\item It is important that before executing the test case you add the files
		to the source code repository using {\tt git add}. Since we don't want to
		store the reference files within the git repository, adding the test case
		at this point will simplify your life. 
	
	\item Use the integration test script, by providing the test case path as
		input, to check whether your test case is working as expected.

	\item If the test case is within a folder containing a \file{test.cfg} file,
		add the name of the test case (which is the same as the folder) to that
		file.

	\item Commit and push the changes. 

\end{itemize}


\subsection{Architecture of the Integration test script}
\todo{26/06}
\subsubsection{Defining a new Pass}
\todo{26/06}

\section{Machines}
