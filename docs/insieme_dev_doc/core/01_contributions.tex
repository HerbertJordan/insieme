\subsection{The Core's Contributions}
\subsubsection{Representation}
The compilers intermediate representation is a formal, Turing-complete,
high level programming language specified within the Insieme IR
Language Specification \cite{insieme_ir_spec}. Internally, the various
constructs required to represent programs are modeled using nodes within a DAG 
(see \ref{sec:Compiler.Core.NodesAndManagers}). Each node has a certain type
(e.g. a variable, a for-loop or an array-type) and a list of child-nodes
representing parameters of the individual node types (e.g. value type, loop
body or length or an array). Although virtually representing a tree, the nodes
are physically organized withing a DAG since many sub-trees within a program
representation (e.g. reused struct types or functions) are identical. By using a
DAG structure, these unnecessary redundancy can be avoided by sharing common
sub-trees. However, as a consequence of the sharing, all nodes within the DAG
have to be immutable.

\subsubsection{Construction}
Every data structure needs to be constructed at some point. All nodes of an IR
DAG can be constructed using static factories associated to the corresponding
node types (see \ref{sec:Compiler.Core.NodesAndManagers.Factories}). In many
cases, this is a very cumbersome way of constructing IR codes since various
frequently used constructs involve the construction of several nodes.
Furthermore, proper typing of expressions is required. Therefore, a builder has
been implemented supporting the construction of typical, frequently used IR
fragments. Its internal organization is covered within section
\ref{sec:Compiler.Core.Builder}. However, for more extensive code sections like
a simple loop nest, using the builder still requires a considerable amount of
coding effort making the resulting code unreadable. Also, it is sometimes next
to impossible to deduce from the code what has been constructed. Therefore, an
\textit{IR Parser} converting a string-based input IR code is provided to offer
a convenient mean to build larger IR constructs. For details see
\ref{sec:Compiler.Core.Parser}.

\subsubsection{Addressing}
Nodes within the IR of a program can be addressed using two distinct means. The
simple one is by obtaining a pointer to the corresponding node. However, in some
cases this is not sufficient, since a pointer to a node within a DAG does not
allow to distinguish multiple usages of the same, shared structure. For
instance, a pointer to a variable node is simply referencing this variable.
Since the same node is commonly referenced by any location the variable is
touched by, a pointer does not allow to determine which actual ``position'' in a
program is to be addressed. However, pointers are a cheap way of addressing
elements and in many cases the actual context is not important. If, however, the
context gets important, \textit{NodeAddresses} can be used. Unlike pointers,
addresses model the full path from some root node (not necessarily the global
root node) to an addressed node within the virtual IR tree. Therefore, all the
context information is available. Addresses may for instance be used when aiming
for replacing specific sections of a program -- sections which might be present
in an identical form within the program at different places. More regarding the
implementation of addressing means in the IR is covered within the corresponding
section \ref{sec:Compiler.Core.PointersAndAddresses}. Section
\ref{sec:Compiler.Core.Visitors} lists the various means offered to iterate over
(sub-sections) of an IR tree using \textit{Visitors} while section
\ref{sec:Compiler.Core.Mappers} covers the basic mechanism for ``mutating'' IR
constructs based on Node \textit{Mappers}.

\subsubsection{Extensions}
Not every construct defined by the INSPIRE language is converted into a
specialized IR node. For instance, basic types (integers,
floats, specialized arrays, \ldots) and a list of operations manipulating
values of those types (addition, array accesses, \ldots) are integrated using a
standardised procedure. The implementation of the procedure used for defining
the core language (also known as Lang-Basic) and various, special-purpose
language extensions are covered within section
\ref{sec:Compiler.Core.LangBasic}.

\subsubsection{IO}
Section \ref{sec:Compiler.Core.Printer} describes the implementation of the IR
pretty printer producing ``human readable'' yet not parse-able output intended
for more effective debugging. The IR Dumpers covered within section
\ref{sec:Compiler.Core.Dumpers} however are capable of storing and loading IR
codes using binary or text-based formats.

\subsubsection{Checks}
In addition to the means representing, printing and modifying applications,
utilities enabling to verify the proper composition of IR codes are offered. The
most essential contribution thereby is the type deduction mechanism covered
within section \ref{sec:Compiler.Core.TypeDeduction}. This mechanism is used
to automatically deduce the type of an expression as well as to verify whether a
given type is matching an expression. It is therefore the foundation for a
series of additional semantic constraints imposed on IR codes and verified by
the checks covered within section \ref{sec:Compiler.Core.SemanticChecks}. 

\subsubsection{Analysing}
A program representation is of no big use if it cannot be investigated. Visitors
provide a very flexible mean to do so, yet they are rather low level constructs.
Although a large number of frequently occurring questions (e.g. is a given type
a reference type? What type is it referencing? Is type A a specialization of
type B?) can be implemented using visitors within a view lines of code, their
application results in unnecessary extensive code sections. Therefore,
frequently required analysing utilities are collected and aggregated within this
sub-module and documented within section \ref{sec:Compiler.Core.Analysis}.
Further, section \ref{sec:Compiler.Core.Statistics} covers utilities implemented
for collecting statistical information regarding the size and depth of an IR
tree, sharing parameters, memory consumptions and node-type distributions.

\subsubsection{Manipulation}
As Visitors provide a foundation for all kind of analysis, Mappers are the
primitive used for altering IR codes. However, du to their additional required
flexibility Mappers are even more cumbersome to employ for particular purposes.
Therefore, basic operations like adding, removing or replacing nodes within an
IR, updating a variable type, inlining, outlining and variable substations are
offered to provide easy-to-use manipulation utilities. The corresponding means
are covered within section \ref{sec:Compiler.Core.Manipulation}.

\subsubsection{High-Level Manipulation}
In some cases thinking about IR constructs using more abstract terms is
simplifying the interaction. For instance, when manipulating arithmetic
expressions common operations like the +, -, * and / should be supported.
Therefore an infrastructure enabling the interpretation of IR constructs for
specific domains has been established. For once, this should simplify
interaction. However, it should also provide incentive to represent information
e.g. obtained by analysis using the already present IR constructs. This was, the
IR is naturally extended to represent any kind of value. The benefit is, that
present utilities can be used for the manipulation of this data (e.g. IO
utilities) and the results can be easily integrated into the output code. The
corresponding means are covered within section \ref{sec:Compiler.Core.Encoding}.
Furthermore, section \ref{sec:Compiler.Core.Arithmetic} covers the the
application of this concepts regarding (partial) arithmetic formulas
encountered frequently within numerical applications.
