\subsection{Of Pointers and Addresses}
\label{sec:Compiler.Core.PointersAndAddresses}

The core offers two means to address IR DAG structures. On the one hand,
\textit{Node Pointers} may be used to reference a single node within the DAG.
Since every node is the root of a virtual IR tree, pointers can just as well be
used to refer to IR code (sub-)fragments. \textit{Node Addresses} on the other
hand are addressing nodes being located relative to a certain ``root'' node. For
instance, an address may address the 2nd child of the 3rd child of the 1st child
of a \textit{ForStmt}. The ``root'' not necessarily need to be the root of the
full program. Relative addresses are fine too.


While pointers are sufficient for almost all applications, in some cases
addresses are required to resolve ambiguities caused by the implicit node
sharing. For instance, consider the IR structure used to model the simple
expression \lstinline|1+1|. There is a node representing the 1, including
another node for its type, and a node representing the + operation. The latter
has 2 child node references, the operands, which in this case will both point to
the node representing the 1. Imagine you want to refer to the first operand
only, for instance, to replace it by a node representing 2. Using a pointer does
not help, since a pointer referencing the first operand is also referencing the
second argument (right, due to the implicit node sharing). To resolve this
problem, addresses are offered. By enabling the user to address a node by
following a path within the ``virtual'' IR tree, ambiguities can be resolved.


Table \ref{tab:Compiler.Core.ptr.a.addr} summarizes the differences and
should provide some hints on when to use what.

\begin{threeparttable}[1h]
	\centering
	\begin{tabular}{c|c|c}
		\textbf{Aspect} & \textbf{Pointer} & \textbf{Address} \\
		\hline \hline
		speed & fast as a plain pointer    & slow \\
		space & \lstinline|sizeof(int*)|   & \lstinline|3 * sizeof(int*)| \\
		thread save & yes & yes/no\tnote{1} \\
		\hline
		equality & address of targeted node & root address + rel. path\tnote{2}
		\\
		\hline
		access node and sub-nodes  & yes & yes \\
		implicit up-cast & yes & yes \\
		typesafe casts & yes & yes \\
		typesafe navigation & yes & yes \\
		integrated in visitor & yes & yes \\
		navigation to parent & no\tnote{3} & yes, up to root\tnote{4} \\
		call context information & no & yes, up to root \\
		annotateable & no & no \\
		\hline
		\multicolumn{3}{c}{} \\
	\end{tabular}
	\begin{tablenotes}
		\item[1] Thread save if constructed directly or using non-shared source
		address
		\item[2] Note that different addresses to the same shared nodes are not
		equivalent
		\item[3] There is no well defined parent for a shared node. You
		should consider the faster possibility of stopping a decent one level earlier
		using pointers instead of using addresses to get back one step
		\item[4]According to the unique list of ancestors specified by the address
	\end{tablenotes}
	\caption{Properties of Pointers and Addresses}
	\label{tab:Compiler.Core.ptr.a.addr}
\end{threeparttable}


\subsubsection{Implementation}
One of the major design goals obeyed by the implementation of pointers and
addresses is to allow provide equivalent syntactical interfaces such that
pointers and addresses can be used interchangeable within generic
implementations (e.g. the visitors \ref{sec:Compiler.Core.Visitors} or the
pattern matcher \ref{sec:Compiler.Transform.Pattern}).

Both, pointers and addresses are generic classes. The generic parameter is
thereby defining the type of node they are referring to. For instance, a
\lstinline|Pointer<const CallExpr>| is referencing a constant call expression
node. However, since using those extensive generic type names is rather
cumbersome, aliases are defined for each node type within
\textit{forward\_decls.h}. The type \texttt{CallExprPtr} can therefore be used
as a replacement for the full name. The address counterpart would be
\texttt{CallExprAddress}.


\paragraph{Files}
The following files are containing the definitions of pointer and address
related constructs and utilities:
\begin{itemize}
  \item \textit{core/ir\_pointer.h} \ldots implementation of the generic
  \texttt{Pointer} class and operators including static and dynamic casts
  \item \textit{core/ir\_address.h} \ldots implementation of the generic
  \texttt{Pointer} class and operators including static and dynamic casts
  \item \textit{core/forward\_decls.h} \ldots type definitions of specialized
  \texttt{Pointers} and \texttt{Addresses} using the \textit{nodes.def} X-macro
  file; additional abbreviations of frequently used types (e.g.
  \texttt{NodeList})
  \item \textit{core/ir\_node\_accessor.h} \ldots defines a generic base class
  for node accessor implementations as well as partial template specializations
  for pointer and address based variations
  \item \textit{utils/pointer.h} \ldots implements the base class the IR
  pointer is derived from as well as a set of operations
  \item \textit{utils/set\_utils.h} \ldots defines a hash-based generic set
  implementation (\texttt{PointerSet}) capable of managing pointers as elements;
  it covers plain as well as smart pointers
  \item \textit{utils/map\_utils.h} \ldots defines a hash-based generic map
  implementation (\texttt{PointerMap}) capable of managing pointers as elements;
  it covers plain as well as smart pointers
\end{itemize}

\paragraph{Accessors}
As has been covered within the node section
\ref{sec:Compiler.Core.NodesAndManagers}, the actual accessor functions allowing
to investigate IR nodes are defined within separated, generic Accessor classes.
These classes are also inherited by the generic \texttt{Pointer} and
\texttt{Address} classes. This way, all accessor functions are inherited.
However, the generic nature fo the accessor infrastructure allows the pointer
class to return pointer types upon member accesses while accessor functions
inherited by the addresses class return extended versions of them self,
referencing to the corresponding child nodes.

Note, the \lstinline|->| operator of pointers and addresses are overloaded to
return pointers to the pointer / address itself. Since the pointer / address is
inheriting the access functions forwarding requests to the actual node, this is
what is according to the users intention.

\subsubsection{HowTo}
Finally, some common problems and solutions related to Pointers and Addresses.

\paragraph{How do I cast a Pointer / Address?}
When casting pointers, up- and downcasting has to be distinguished. Upcasting is
supported automatically by pointers and addresses. Hence, constructs like
\begin{lstlisting}
	StatementAddress a = someSource(..);
	// here a is automatically up-casted
	NodeAddress b = a;
	// also here
	b = a;
\end{lstlisting}

However, since the down-cast is not generally safe, down-casts are note
conducted implicitly. As for plain and smart pointers, static and dynamic casts
are offered. Static casts are faster, yet require the programmer to ensure that
the cast is valid. dynamic casts are slower since they are verifying the correct
type during runtime. Note: in debug-mode, the current implementation of the
static casts is verifying the proper typing as well. If incorrect, an assertion
is triggered.

An up-cast can be conducted as follows:
\begin{lstlisting}
	NodePtr a = someSource(..);
	StatementPtr b = static_pointer_cast<StatementPtr>(a);
	StatementPtr c = dynamic_pointer_cast<StatementPtr>(a);

	NodeAddress d = someSource(..);
	StatementAddress e = static_address_cast<StatementAddress>(d);
	StatementAddress f = dynamic_address_cast<StatementAddress>(d);
\end{lstlisting}

Be aware, while the static cast is returning a re-typed reference to the
original pointer or address, the dynamic cast returns a fresh, re-typed copy.

Since in many cases, where the type is clear, the syntax of the static cast is
quite extensive, both, pointers and addresses offer a short-cut. The \texttt{as}
member function can be used to quickly conduct a static cast.

\begin{lstlisting}
	StatementPtr a = someSource(..).as<StatementPtr>();
	StatementAddress b = someSource(..).as<StatementAddress>();
\end{lstlisting}

\paragraph{How do I check for Null?}
Pointer and Addresses can be Null (not referencing any element) and
bool-convertible. Hence, any pointer / address can be used as a boolean
expression to test whether they are referencing a Null value.

\paragraph{How do I obtain an Address from a Pointer?}
A pointer can be converted into an Address using the corresponding construction.
To avoid accidental conversion, the constructor is marked \lstinline|explicit|,
hence has to be invoked manually. Example:
\begin{lstlisting}
	ExpressionPtr a = someSource(..);
	// implicit
	StatementAddress b(a);
	// explicit
	b = StatementAddress(a);
\end{lstlisting}
Up-casts are thereby supported. The resulting address will use the given pointer
as its root node. Further, the length of the address is 0, hence, it is
addressing the root node itself.

\paragraph{How do I obtain a Pointer from an Address?}
The conversion from an address to a pointer referencing its referenced node is
implicit. Hence, a simple assignment is sufficient.
\begin{lstlisting}
	ExpressionAddress a = someSource(..);
	StatementPtr b = a;
\end{lstlisting}
Alternatively, the member function \lstinline|getAddressedNode()| can be used.
To obtain a pointer to one of the parent nodes \lstinline|getParentNode()| can
be used. To obtain the address of a parent node, use
\lstinline|getParentAddress()|.

\paragraph{How do I create Pointers?}

\paragraph{How do I construct Addresses?}

\paragraph{How do I concatenate Addresses?}

\paragraph{How do I create a Set of NodePtrs / NodeAddresses?}



