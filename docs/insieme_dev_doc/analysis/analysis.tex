\chapter{The Analysis Module}

\section{The Polyhedral Model [Simone]}
\label{insieme:analysis:polyhedral}

\subsection{Overview of the Polyhedral Model}

\begin{figure}[t]
\small
\begin{lstlisting}[caption=Running example]
for(unsigned i=0; i<n; ++i) {
	A[i][i] = 0;												S0
	for(unsigned j=i; j<n; ++j) { 
		A[i][j+1] = i+j;				 					    S1
		(i+j)%2 && (A[n-j][n-i] = j-i);	         		        S2
	}
}
\end{lstlisting}
\label{list:1}
\end{figure}

The polyhedral model represents, in an algebraic way, the execution of a
program.  It captures both the control-flow and data-flow of a program using
three compact linear algebraic structures, described in the following
subsections. The main idea is to define, for a statement $S$, a \emph{space} in
$Z^n$ where each point correspond to an execution, or \emph{instance}, of $S$.
The value of the coordinates of a point within this space represents the value of
the $N$ loop iterators spawning statement $S$.  In order to keep the
representation compact, such space, called \emph{polyhedron}, is defined by a
set of bounding \emph{affine hyperplanes}.
%end not by enumerating the single points.

\begin{definition}
\label{def:1}
(Affine Hyperplane). The set X of all vectors $x \in \mathbb{Z}^n$ such that $h.\vec{x} =
k$, for $k \in \mathbb{Z}$.
\end{definition}

\begin{definition}
\label{def:2}
(Polyhedron). The set of all vectors $\vec{v} \in \mathbb{Z}^n$ such that $A \vec{v} +
\vec{b} \ge 0$, where $A$ is an integer matrix.
\end{definition}

\subsubsection{Iteration Domain}
Because the space is bounded by a set of all affine inequalities the
corresponding integer polyhedron is convex. A bounded polyhedron is also called
\emph{polytope}. The space on which a statement is define is also referred to as
its \emph{Iteration Domain}, $\mathcal{D}_S$. For example lets consider the code in
Listing~\ref{list:1}. This loop nest contains 3 statements which are referred to
as $S0$, $S1$ and $S2$. Each statement is defined within an iteration domain
which is bound by the surrounding control flow statements. For example the
iteration domain for $S0$, $S1$ and $S2$ is defined as follows:
\begin{align*}
	\mathcal{D}_{S_0} = & \{~i~|~0 \le i < n \} \\
	\mathcal{D}_{S_1} = & \{~i,j~|~ 0 \le i < n \wedge i \le j < n \} \\
	\mathcal{D}_{S_2} = & \{~i,j~|~ 0 \le i < n \wedge i \le j < n ~\wedge  
						\exists~e \in \mathbb{Z}~|~ i+j-2e=1 \}
\end{align*}

As described in \ref{def:2}, iteration domains are represented by the integer
matrix $A$ multiplied by a so called \emph{iteration vector} $\vec{x}$. The
iteration vector determine the dimensionality of the space on which a statement
is defined therefore composed by the loop iterators enclosing a statement. For
example iteration domain for statement $S1$ in listing~\ref{list:1}~is defined
by the vector $\vec{x_{S_1}} = \begin{pmatrix} i \\ j \\ n\end{pmatrix}$.
$i$ and $j$ are said iterators (since they vary within the defining loop
boundaries) while $n$ is a parameter. 
Conventionally the matrix $A$ is represented using a so-called
\emph{homogeneous} coordinates so that vector $\vec{b}$ is added as its last
column. Iteration domain for statement $S_1$ is therefore represented as
follows:

\begin{align*}
\mathcal{D}_{S_1} = & \begin{Bmatrix} 
	\begin{pmatrix} i \\ j \end{pmatrix} \big|
	\begin{pmatrix} i \\ j \end{pmatrix} \in \mathbb{Z}^2,
	\begin{bmatrix} 1 & 0 & 0 & 0\\ -1 & 0 & 1 & -1 \\ 
			-1 & 1 & 0 & 0 \\  0 & -1 & 1 & -1 
	\end{bmatrix} \cdot
	\begin{pmatrix} i \\ j \\ n \\ 1 \end{pmatrix}
	\ge \vec{0}
\end{Bmatrix}
\end{align*}

\subsubsection{Scheduling Function}
The second piece of information which is required to describe the semantics of a
program are the so-called \emph{scheduling} (or \emph{scattering})
\emph{functions}. Intuitively, statements belonging to a loop body, and subject
to the same control flow, will share identical iteration domains. The
information of the order on which statement instances are executed is not
represented. A \emph{schedule}, $\theta(\vec{x})$, is a function which
associates a logical \emph{execution date} to each instance of a statement. This
allows the ordering of the instances defined by the iteration domain and
furthermore it defines an execution order for instances of different statements.
A schedule $\theta(\vec{x})$ has the following shape:
\[
\theta_S (\vec{x}) = T_S\vec{x} + \vec{t_S}
\]
where $\vec{x}$ is the iteration vector, $T_S$ is an integer constant
transformation matrix and $\vec{t_S}$ is a constant vector. $T_S$ and
$\vec{t_S}$ can be merged together into a matrix $\mathcal{S}$ if the system is
represented based on homogeneous coordinates. 

A standard and simple way of assigning scheduling functions to statements of
program is done on the basis of the AST as explained in the Cloog user
guide~\cite{cloog}.
For example scheduling functions
for statements $S0$, $S1$ and $S2$ of code in listing~\ref{list:1}~are defined
as follows:
\begin{align*}
\mathcal{S}_{S_{0}} = \begin{bmatrix} 1 & 0 \\ 0 & 0 \end{bmatrix} 
					\begin{pmatrix} i \\ 1 \end{pmatrix} 
\end{align*}

\begin{align*}
\mathcal{S}_{S_{1}} = \begin{bmatrix} 1 & 0 & 0 \\ 0 & 0 & 1 \\ 
 										0 & 1 & 0 \\ 0 & 0 & 0
 					\end{bmatrix} 
 					\begin{pmatrix} i \\ j \\ 1 \end{pmatrix}  
\end{align*}

\begin{align*}
 \mathcal{S}_{S_{2}} = \begin{bmatrix} 1 & 0 & 0 \\ 0 & 0 & 1 \\ 
 										0 & 1 & 0 \\  0 & 0 & 1 
 					\end{bmatrix} 
					\begin{pmatrix} i \\ j \\ 1 \end{pmatrix}
\end{align*}

Intersection of the scheduling function by the corresponding iteration domain of
the statement produces a sequence of tuples, or logic dates, representing
the execution order of each statement instance. Lexicographically 
ordering the set will give the exact sequence of statements instances executed
by the program. % oh god this is ugly
\begin{align*}
	S_0:(0,0) & \prec S_1:(0,1,0,0) \prec S_2:(0,1,0,1) \prec \\ 
			  & S_1:(0,1,1,0) \prec S_2:(0,1,1,1) \prec \ldots \prec \\
	S_0:(1,0) & \prec S_1:(1,1,0,0) \prec S_2:(1,1,0,1) \prec ...
\end{align*}


\subsubsection{Access Function}
One last function is also required to capture the data locations on which a
statement operates. The \emph{access} (or \emph{subscript}) \emph{function} 
describes the index expression utilized to access arrays, and therefore memory
locations, within a statement. Representation of access functions is similar to
what already described for scattering functions. For example array 
access in statement $S1$, \texttt{A[i][j+1]}, can be represented in matrix form
with one row for each dimension being accessed:
\[
\mathcal{A}_{USE(A)} =  \begin{bmatrix} 1 & 0 & 0 \\ 0 & 1 & 1 \\ 
						\end{bmatrix} 
						\begin{pmatrix} i \\ j \\ 1 \end{pmatrix}
\]
Access functions also store the information whether a particular memory location
is being read (\texttt{USE}) or written (\texttt{DEF}). This kind of
information is utilized by the polyhedral model to compute exact dataflow and
dependency analysis for a given code region. 

\subsubsection{Static Control Part (SCoP)}

The three functions described above can completely describe (both semantically
and syntactically) a code region (also called SCoP) which respects the
constraints of the Polyhedral Model. A SCoP is defined to be the maximal set of
consecutive instructions such that: loop bounds, conditionals and subscript
expression are all affine functions of the surrounding loop iterators and global
variables; loop iterators and global variables cannot be modified. 

By definition, a subset of a SCoP is still a SCoP. A program usually may contain
several non-overlapping SCoPs. 

\subsection{Polyhedral Model Data Structures in Insieme}

The support for the PM is included in the \file{analysis/polyhedral} and
\file{transform/polyhedral}. In this Section only the analysis part is covered. 

\subsubsection{The \type{IterationVector}}

The \file{analysis/polyhedral/iter\_vec.h} abstracts the concept of iteration
vector covered above. An iteration vector is composed by a list of
\type{Iterators}s, a list of \type{Parameter}s and a \type{Constant} part.
Implementation is quite straightforward. 

Important to notice that while iterators must be an IR \type{Variable}, a
parameter can be any \type{Expression} (this is needed to be able to represent
struct members as parameter of a SCoP). An iteration vector can be either build
with a predefined layout or manipulated by adding components. An empty iteration
vector always have at least a constant part. 

\begin{srcCode}
IterationVector iterVec(vector<Iterator>{v1,v2}, vector<Parameter>{v3});
out << iterVec; // (v1,v2|v3|1)

// or 

IterationVector iterVec;
iterVec.add(Iterator(v1));
iterVec.add(Iterator(v2));
iterVec.add(Parameter(v3));
out << iterVec; // (v1,v2|v3|1)
\end{srcCode}

Printing an iteration vector to an output stream produces a representation which
lists iterators followed by parameters and the constant part 1. Each part is
separated by a {\tt |}. 

Methods are included which iterates through the elements of an iteration vector,
or a subset of them (iterators and parameters). The number of iterators is given
by the {\tt getIteratorNum()} function while the number of parameters is given
by {\tt getParameterNum()}. The index of an element is given by the {\tt
getIdx()} function which returns -1 if the variable being searched is not in the
iteration vector. 

For additional information on how to manipulate an iteration vector refer to the
unit test file \file{analysis/test/polyhedral/polyhedral\_test.cc}. 

In the \file{analysis/polyhedral/iter\_vec.h} file several utility functions are
provided to work with iteration vectors. The {\tt IterationVector merge(const
IterationVector\& iv1, const IterationVector\& iv2)} function in particular is
useful when an iteration vector needs to be generated from two distinct vectors.
The operation simply creates a new iteration vector containing the iterators of
vector {\tt v1} before the ones of vector {\tt v2} and the same pattern for
parameters. A second function is the {\tt transform} utility. This function is
used to create a map which transforms a functions which are expressed in terms
of one iteration vector into another. The operation is valid only if the source
vector is smaller than the target and if all the elements present in the source
appear as well in the target vector. Use of this transformation map
(\type{IndexTransMap}) will be explained in the next section.

\subsubsection{The \type{AffineFunction}}

An affine function is an expression which of the form $c_1 \cdot a_1 + c_2 \cdot
a_2 + \ldots + c_n \cdot a_n + c_0 \cdot 1$ where $c_i \in \mathbb{Z}$ and $a_i
\in \mathbb{V}ar$. The concept of an affine function is represented in the file
\file{analysis/polyhedral/affine\_func.h} by the
\type{analysis::polyhedral::AffineFunction} class. 

The representation uses a compact representation (in order to limit the space
used by the PM representation), it stores a reference to an iteration vector
which contains the variables (i.e. $a_i$) of the affine function, a vector of
integer coefficients ({\tt coeffs}) $c_i$, and an integer value ({\tt sep})
which dictates the index of the first coefficient (in {\tt coeffs}) which refers
to parameters. For the iterators or parameters for which no coefficient is
provided, a 0 value is assumed.

For example given the iteration vector $(v_1,v_2|v_3|1)$ and the coefficient
vector $\{1,3\}$ and $sep = 1$; this corresponds to the affine function: $v_1 +
3 \cdot v_3$ since coefficient for $v_2$ and the constant part are not provided
and therefore automatically set to 0. Instead, if the coefficient vector is
$\{2,3,4,5\}$ then $sep = 2$, the relative affine function is $2 \cdot v_1 + 3
\cdot v_2 + 4 \cdot v_3 + 5$. 

An affine function object can be build from an expression or a \type{Formula}
object.  In this case the iteration vector being passed to the constructor is
updated with the iterators and parameters found within the expression. New
elements of an iteration vector are always added in append (iterators are
appended to the list of iterators and parameters append to the list of
parameters). In this way affine functions referring to the same iteration vector
remain valid. Once built, the iteration vector associated to an affine function
cannot be modified. {\bf NOTE:} It is responsibility of the programmer to make
sure that the iteration vector does not out-live the affine function instance.

A member function is also available to ``move'' an affine function to another base
(or iteration vector) {\tt toBase(const IterationVector\&)}. This operation may
fail if the new base is not compatible with the old. In order to speedup the
process of converting many affine function to a new base the {\tt transform}
method defined in \file{analysis/polyhedral/iter\_vec.h} allows the creation of
a transformation map which can be applied to an affine function to change it
into a new base. 

For further details related to the use and manipulation of affine functions
refer to the test units in \file{analysis/test/polyhedral/polyhedral\_test.cc}. 

\subsubsection{The \type{Constraint} and \type{AffineConstraint}} 

The \type{insieme::utils::Constraint} is a generic class used to represent
constraints of the type $f(x) == | != | < | <= | > | >= 0$. Where $f(x)$ can be
any object which represent a function. Since it is generic with respect of the
functor being used (e.g. \type{Constraint<Formula>} or
\type{Constraint<Expression>} can be easily instantiated and used) its
definition is within the utilities of Insieme,
\file{insieme/utils/constraint.h}. In the case of the polyhedral model the
template class is instantiated as \type{Constraint<AffineFunction>}.  Therefore
it encodes constraints of affine functions.  The typename \type{AffineFunction}
is introduced as an alias for such instantiation. 

Constraints can be combined to create more complex predicates. This is obtained
through the \type{Combiner<T>} class. The Combiner is an {\em abstract} class
which is then realized through three concrete classes:

\begin{description}

	\item[\type{RawConstraint<T>}:] This is a wrapper for a plain constraint
		object. 

	\item[\type{NegConstraint<T>}:] It represent the negation of a sub-constraint
		which can be either a plain constraint or a composition of two or more
		other constraints. 

	\item[\type{BinConstraint<T>}:] This combiner is used to represent either
		conjunction, i.e. $C_i \wedge C_j$, or disjunction, i.e. $C_i \vee C_j$,
		of two constraints. 

\end{description}

The representation is based on the {\em composite} design pattern, therefore
this allows complex structures to be represented as a tree. One of the key
features of this representation is that beside storing the semantics of a
constraint it also keeps its syntax (or aspect) unvaried. 

Combining constraints is simplified by a set of operations which have been
overloaded to simplify the construction of complex expressions. 

\begin{description}

	\item[{\tt \&\&}:] The {\em logic and} operator is used to build a conjunction
		between two constraints. 

	\item[{\tt ||}:] The {\em logic or} operator is used to build a disjunction
		between two constraints. 

	\item[{\tt \textasciitilde{}}:] The not operator is used to negate a constraint. 

\end{description}

For actual examples of how Constraints objects are created and manipulated
please refer to unit test cases in
\file{analysis/test/polyhedral/polyhedral\_test.cc}. 


In order to explore the content of a constraint (since it is an expression tree) the visitor
pattern is utilized. The class \type{ConstraintVisitor<T>} is responsible for
that providing an easy way to scan a constraint for particular properties. A
derived class offers a basic visitor with recurring semantics, the
\type{RecConstraintVisitor<T>}. An example of how the recursive visitor can be
used is represented by the \type{ConstraintPrinter<T>} class which can be used
to print to standard output any constraint. 

Several utilities are also included for constraints. For example the {\tt toDNF}
function is responsible to transform a generic constraint into a disjunctive
normal form (DNF). This part needs some refactoring!

\subsubsection{The \type{AffineSystem}}

The abstraction of an affine system is used to represent a set of constraints.
Since in the PM it is common to assume the constraints are in the form $ F \ge
0$ or $F == 0$ (depending on where the system is being used), an affine system
can be represented by only storing the set of affine functions.  The
\type{AffineSystem} defined in the file \file{analysis/aff\_sys.h} has this
goal. It is important that all functions within a system are based on the same
iteration vector, and this is enforced during the construction of an instance of
the \type{AffineSystem} class. 

Given an iteration vector, an affine system can also be built by providing a
coefficient matrix. Additionally a new constraint can be add by providing a
coefficient vectors which refers to the new constraint. Given an
\type{AffineSystem}, the coefficient matrix can be extracted with the {\tt
extractFrom} function. 

An \type{AffineSystem} is utilized to represent the matrices associated with the
polyhedral model representation. In particular scheduling and access functions
are represented in term of an affine system. 


\subsection{SCoP Analysis}

Given an IR node (which can be any of the IR structures) SCoP analysis can be
invoked to determine whether the subtree (spawning from that statement) is a
valid SCoP (i.e. a code region which can be represented in terms of the
Polyhedral model constraints). The entry point of the SCoP analysis is in the
file \file{analysis/polyhedral/scop.h}. The method \type{AddresSList mark(const
NodePtr\& root)} takes any IR node and return a list of addresses (having {\tt
root} as a root node) which are the top level entry points of SCoPs within {\tt
root}. By definition SCoPs are never overlapping. 

Given the entry-point statement of a SCoP the \type{ScopRegion} annotation can be
checked. This will return an object of type \type{Scop} which contains (cached)
all the information necessary to represent the code region in terms of the
polyhedral model. 

Given a statement it is also possible to retrieve the \type{Scop} object using
the utility function \type{optional<Scop> toScop(const NodePtr\&)}.

\subsubsection{SCoP Internal Representation}
As said before, representing a code with the Polyhedral Model requires three
data structures to be stored. In order to maintain the connection between the
polytope and the code we use IR annotations to store iteration domains,
scheduling and access functions. 

The main issue to understand here is that usually SCoP analysis finds largest
code regions which can be represented in term of the PM. This makes sense since
the more code in the region, the better understanding of the code dependencies
the analysis can gain and the more opportunities for optimizations. However the
user may ask to apply a specific optimization starting from a sub portion of a
SCoP. In Insieme this is supported thanks to the immutable nature of the IR
nodes. 

The SCoP analysis is done in a bottom-up fashion. Since the nodes are shared the
polyhedral model representation of a node is stored in a context-insensitive
way. This means that the polyhedral representation at a specific node only
includes information within the child nodes. For example, the statement {\tt S0}
{\tt A[i+n+1]=0} generates an iteration vector of the form $v_{S0} = (|i,n|1)$.
It is worth noting that at this level we have no knowledge of whether variables
$i$ and $n$ are loop iterators or parameters of the model. This iteration vector
will be stored in the \type{ScopRegion} object which is associated to this
statement.  Also the access function affine system is stored within the
\type{ScopRegion} annotation.  Let now consider a possible parent statement node
{\tt S1} of the form {\tt for(int i: 0..10) \{ A[i+n+1] = 0; \}}.  This
statement will be characterized by the iteration vector $v_{S1} = (i|n|1)$ which
states that $i$ is an iterator while $n$ is a parameter. Iteration vector
$v_{S1}$ will be stored to the \type{ScopRegion} instance attached to {\tt S1}.
Additionally, since {\tt S1} is a control statement the information of the
domain it spawns are stored. The Scop annotation of {\tt S1} will therefore
carry an iteration domain object of the form $i>=0 \wedge i<10$. 

Whenever a statement is analyzed we first check whether the statement already
has a \type{ScopRegion} annotation attached. If not the analysis is performed on
the statement. Whenever the all the children of a statement have been analyzed,
we merge the iteration vectors into a new iteration vector which is then
utilized to describe accesses and domain functions within the parent node. This
way of building up the SCoP representation allows to reduce the amount of
computation needed if otherwise the nodes were not shared. 

During this pass one things is checked, that every access function and loop
bound and control-flow expression is an affine linear function. If not an
exception is thrown and nodes are marked as not SCoPs. After SCoP regions are
formed (so that we know which of the variables within the SCoP are iterators and
which ones are parameters), an analysis based on the information stored on the
SCoP annotations is performed to make sure that within a SCoP no parameter value
is modified and that iterators are only modified within loop statements. 
This task is performed by the \type{postProcessSCoP} function in the file
\file{analysis/scop.cpp}.

\subsubsection{Instantiation of a SCoP}
As we said in the previous section, each statement within a SCoP carries
information on the iteration vector, domain and access functions. Children of a
statement however have their information stored based on different iteration
vectors (since a parameter being used in the then section of a if statement will
not appear between the parameters of the else part of the statement). In order to
build the polyhedral representation of a region of code all access functions and
domain expression have to be based on the same iteration vector. 

The \type{polyhedral/Scop} class, defined in
\file{analysis/polyhedral/polyhedral.h} has the goal to fully represent a SCoP.
The process works as follows. The user addresses a statement {\tt S} which is either the
entry of a SCoP or within the SCoP. From {\tt S} he can decide to
generate the Polyhedral representation of the code region derived from this
statement. This process build a \type{Scop} object where the affine expressions have
been updated to the new base (which is the iteration vector, $vec_S$). This
operation is implemented by the \type{resolveScop} function defined in
\file{analysis/polyhedrl/scop.cpp}. 

\section{The Control Flow Graph [Simone]}
\label{insieme:analysis:cfg}



\section{Dataflow Analaysis [Simone]}
\label{insieme:analysis:dtaflow}
