\chapter{The Insieme Compiler} \label{cap:compiler}
\index{Compiler}

\section{Overview}
\subsection{Architecture}
\subsection{Coding Standards}
\subsection{Source Code Organization}
\subsection{Building a Simple Optimizing Compiler}
\label{cap:compiler:sec:overview:sub:building} TODO: add an example loading
code, changing something stupid and generating output code.

\section{The Utilities}
\subsection{Logging}
\subsection{Compiler}
\subsection{Lua}
\subsection{Others}
\subsubsection{Container Utilities}
\subsubsection{Set Utilities}
\subsubsection{Cache Utilities}
\subsubsection{String Utilities}
\subsubsection{Type Traits}

\section{The Core}
The main contribution of the core is to provide data structures to represent
program code using INSPIRE as well as primitives and utilities for inspections,
analysis, modifications and IO.

The basic program code representation is realized using IR nodes (see section
\ref{sec:Compiler.Core.NodesAndManagers}). 



\subsection{The Nodes and their Managers}
\label{sec:Compiler.Core.NodesAndManagers}

To be covered:
\begin{itemize}
  \item Purpose of Nodes - distinction to the IR Spec
  \item Fact that nodes form a DAG
  \item Node Identity specified by type + children or value (little formally)
  \item Node Manager concept for memory management and sharing
  \item Handling the node manager in parallel environments
  \item Implementation of nodes (macros and generics)
  \item List of involved files (nodes.def, ir\_nodes.h, ir\_types.h)
  \item NodeType, NodeCategory, Nodes, Values, Types, Expressions, Statements,
  Support, Program, Accessors, \ldots + relations
  \item How to add new nodes
  \item How to add a method to a node
  \item How to add a field to a node (when to do so - e.g. not if it is
  something identity critical)
\end{itemize}

\subsubsection{Types}
\subsubsection{Statements}
\subsubsection{Expressions}
\subsubsection{Support}

\subsection{Of Pointers and Addresses}
\subsection{Visitors}
\subsection{Mappers}
\subsection{Lang-Basic and Extensions}
\subsection{The Printer}
\subsection{The Dumpers}
\subsection{The Type Deduction}
\subsection{The Semantic Checks}
\subsection{Analysing Utilities}
\subsection{Manipulation Utilities}
\subsection{Arithmetic Utilities}
\subsection{IR Data Encoding}
\subsection{The Parser}

\section{Annotations}

\section{The Frontend}
\subsection{The Frontend's Architecture}
\subsection{The C Converter}
\subsection{The Pragma Parser}
\subsection{The OpenMP Frontend}
\subsection{The OpenCL Frontend}
\subsection{The MPI Frontend ?}
\subsection{The C++ Frontend}
\subsection{Customizing the Frontend}

\section{The Backend}
\subsection{The Backend's Architecture}
\subsection{The Sequential Backend}
\subsection{The Runtime Backend}
\subsection{The OpenCL Kernel Backend}
\subsection{The OpenCL Host Backend}
\subsection{Customizing the Backend}

\section{The Simple Backend}

\section{The Analysis Module}
\subsection{SCoP Analyses}
\subsection{Features}

\section{The Transformation Module}
\subsection{The Transformation Framework}
\subsection{Of Patterns and Generators}
\subsection{The Polyhedral Transformations}
\subsection{The Rule-based Transformations}

\section{The Machine Learning Module}

\section{The Driver}
\subsection{The Measurement and Instrumentation Utilities}
\subsubsection{Remote Execution}
\subsection{The Integration Test Loader}

\section{The XML Dump}
\section{The Playground}
