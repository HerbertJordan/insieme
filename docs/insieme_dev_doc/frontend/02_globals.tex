\subsection{Global/Static Variables [Simone]}
\label{sec:Insieme.Frontend.Global}

One of the main effort of the frontend is to convert the nominal semantics C/C++
input code into structural form which is required by the INSPIRE representation.
Indeed, in C and C++ symbols are addressed by the name and not the symbol
definition itself. This allow to declare a variable, or class, in the header
file and place the corresponding definition in any of the translation units.
However while there can be multiple declaration of a symbol, there must be only
one definition (a.k.a. one-definition-rule). 

In INSPIRE things works differently as the language is based on a structural
form where objects are identified by their structure and not their name. This
produce several implication for example with recursive types and function calls
which we will cover later in Section~\ref{sec:Insieme.Frontend.Recursion}. The
structural nature of Insieme also imposes limitations on several feature of the
C language. One of them is global and static variables. 

\subsubsection{Lack of Global Variables in INSPIRE}
The main reason for not having native support for global variables in Insieme is
the fact that INSPIRE was designed to support parallel analysis and code
transformations. Indeed, having global and static variables forces the
middle-end of the compiler to deal with such concepts thereby adding more
complexity to the analysis and transformation. Fortunately, the practice of
using global variables is highly discouraged and there is always a way to
rewrite a program in order to remove any use of global variables. 




