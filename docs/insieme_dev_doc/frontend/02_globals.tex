\subsection{Global/Static Variables [Luis and Herbert]}
\label{sec:Insieme.Frontend.Global}

Global scope variables are complicated to handle during the translation because the whide scope and
the diferent translation units where the definition might be.\\
To make it a little more complicated, there are several kind of globals and they might produce
errors if not handled nicely.

\subsubsection{previous versions of INSPIRE}

Previous versions of INSPIRE made use of a so called globals elimination algorithm. The idea behind
this procedure was to eliminate the global access to variables and substitute it by an structure
which groups the storage of all those elements, this struct was allocated in main function stack and
passed by reference to all functions which needed it, directly or indirectly. 
Such structure was initialized at the begining of the program.

The main reason for not having native support for global variables in Insieme was
the fact that INSPIRE was designed to support parallel analysis and code
transformations. Indeed, having global and static variables forces the
middle-end of the compiler to deal with such concepts thereby adding more
complexity to the analysis and transformation modules.

This aproach is still valid, but such transformation should be done in the IR and not during
original translation.

\subsubsection{Regarding globals}

Using globals in parallel programs leads to syncronization problems and they my be a source of race
conditions. They should be avoided but reality is that they are widely used by every kind of
programmers. Even new programing languajes for parallel computing count with spetial global
variables (CUDA or openCL).

Related work on why it is important to avoid global variables in parallel programs can be found in
\cite{Zheng:2011:AHG:2117686.2118457}. 

\subsubsection{The Global variable collector}

Globals need to be localizad and counted across different translation units to find out their scope
and visibility.  All translation units are analyzed, for each variable declaration, it is stored. 
All variables declared inside of a function which have global storage are Statics, the rest are
globals. If no definition is found, it should remain Extern.

\subsubsection{Generating a Global Var in IR}

Globals are not definied within the IR, because IR only covers the code inside of the execution
tree. They can be just called by name, making use of a literal node, named as the global, and typed
with the right type.

\textit{lookupVariable} function in basicconvert.cpp takes care of the generation and caching of every
variable used within the program, therefore whis is the right place to build this spetial construct.

\subsubsection{Globals Initialization}

Once whole program has being translated, is time to initialize those globals which have some default
value at their's declaration. All other which have only memory allocation are not needed to be
initialized as the standar dictates. 

At the beginig of the main function, all those globals with initialization are assigned the right
value/expression. Notice that this is an assigment, so the left side must be typed
\textit{ref<type>} while the right side is \textit{type}

\begin{srcCode}
int global = 0;

int main(){
	static int a;  
	a++;
	globalmain++;
	return 0;
}

/////////////////////////////////
// turns to be something like:

let fun001 = fun() -> int<4> {
    global := 0;
	a := CreateStatic(type<int<4>>);

	gen.post.inc(AccessStatic(a));
	gen.post.inc(global);

	return 0;
};

// Inspire Program 
//  Entry Point: 
fun001

\end{srcCode}


\subsubsection{How the backend proccedes [HERBERT]}

\subsubsection{Globals and OpenMP}

All global Variables can be marked \textit{thread private} using an omp pragma. This variables IR
representation needs to be annotated so the OpenMP transformations still know about this issue.

\subsubsection{Globals Relatives}

Globals are present in different flavours:

\textbf{Extern variables}\\
	Those are variables which are declared without definition but can be used in the current
	translation unit. The global collector searches for declarations in all translation units and
	update the storage type to global if the same symbol is defined in other TU. Otherwhise will
	remain extern and it will not be declared extern by the backend so the definition needs to be
	linked by the backend compiler.\\
\textbf{Static variables}\\
	This variables have local visibility within the function where they were declared. They can
	produce aliasing problems because they can have common name with a global. Backend could find
	some trouble to diferenciate those since the name of the symbol is the same in both global and
	function static literal, if the type is also equal, there will be no chance to diferenciate
	them.
	That is why static variable names are modified with an incremental counter, so they turn unique.

	Because of the spetial initialization forced by standard, it is needed to wrap this variables in
	an spetial type which holds the flag for the unique initial initialization. 
	They need to be: created in main (to zero initialize the flag), initialized at the first
	function call (if they have initialization) and unwrapped within every read/writte access (this
	is done automaticaly by \textit{lookupVariable})

\begin{srcCode}

	int f(){
		static a= 0;
		return a++;
	}

	int main(){
		f();
	}

/////////////////////////////////
// turns to be something like:

	let type000 = struct __static_var <
		initialized:bool,
		value:int<4>
	>;

	let type001 = struct __static_var <
		initialized:bool,
		value:'a
	>;

	let fun000 = fun() -> int<4> {
		InitStatic(static_a0, 0);
		return gen.post.inc(AccessStatic(static_a0));
	};

	let fun001 = fun() -> int<4> {
		static_a0 := CreateStatic(type<int<4>>);
		fun000();
	};

\end{srcCode}

\textbf{Global Static}\\

	A global static is a variable which visibility is only the current traslation unit.

	\textbf{wow!} this is not implemented, the problem will only happen if two variables are declared static
	global in two different translation units and they share name (including namespaces if any)

	





