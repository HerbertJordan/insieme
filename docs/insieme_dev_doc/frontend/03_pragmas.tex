\subsection{User Pragmas [Simone]}
\label{sec:Insieme.Frontend.Pragmas}

One of the main features of the Insieme frontend is the ability to easily define
new pragmas. For this purpose a framework has been written which allows the
definition of user pragmas similarly to EBNF form. The framework takes care of
matching pragmas in the input code against the specification given by the user.
If the pragma cannot be matched, an error is produced. Otherwise, if the pragma
is compatible with the EBNF specification, an annotation node is automatically 
generated and associated to the corresponding Clang AST to which the pragma is 
referring. 


\subsubsection{Registering Pragma Handlers}

The implementation of the pragma handling framework is located in the namespace
\decl{frontend::pragma}. The entry point of the framework is the
\type{BasicPragmaHandler<T>} class, define in \file{frontend/pragma/pragma.h},
which is the handler object being registered to the {\tt LLVM/Clang} parser to
be invoked when a pragma is encountered in the input file. In order to
facilitate the creation of such handler objects, the
\type{frontend::pragma::PragmaHandlerFactory} class is defined. 

An example of how a new pragma handler is specified follows:
\begin{srcCode}
clang::PragmaNamespace* omp = new clang::PragmaNamespace("omp");
pp.AddPragmaHandler(omp);

PragmaHandlerFactory::CreatePragmaHandler<OmpPragmaSection>(
	pp.getIdentifierInfo("section"), tok::eod, "omp"
);
\end{srcCode}

First of all {\tt LLVM/Clang} needs a \type{clang::PragmaNamespace} object to be
created which represents the base-name of the pragma. This must be the string
which follows the ``\#pragma'' keyword in the input program. In the example we
define an handler for \srcCodeInl{#pragma omp section}, therefore the namespace
is defined to be \srcCodeInl{omp}. After the namespace is created, we register
it by adding the handler to the current preprocessor ({\tt pp}) which can be
retrieved by the \type{frontend::ClangCompiler} class (note that the Clang
preprocessor shall take ownership of the provided pointer, therefore there is no
need to cleanup the memory, this will be done by Clang once the preprocessor is
destroyed). The code then generates a specific handler for the ``section''
keyword using the \type{PragmaHandlerFactory}. The registration requires the
user to provide the keyword for which this handler must be invoked, an EBNF
specification (which we will explain later), and the name of the namespace
provided as a string. Additionally a template parameter must be provided which
represent the class being instantiated to hold the informations contained on the
matched pragma (\type{OmpPragmaSection} in the example). 

For new pragmas, the user should define a class in order to process and store
the user data contained in the pragma itself. The class
\type{frontend::pragma::Pragma} defined in \file{frontend/pragma/pragma.h}
provide a base class for such purpose. A pragma is defined to store the location
of the start and end location and a reference to the Clang node to which the
pragma was attached. In C a pragma can refer to two kind of nodes, either
declarations (e.g. \type{clang::FuncDecl}, \type{clang::VarDecl}) or a generic
statement. The methods \type{isDecl()} or \type{isStmt()} of the
\type{frontend::pragma::Pragma} class shall be utilized to test the type of the
target node. The methods \type{getDecl()} or \type{getStmt()} can be used to
retrieve a pointer to the target node. 

\subsubsection{Overview of Pragma Matching}

The constructor of the class provided to the \type{CreatePragmaHandler} function
is invoked automatically by the framework once a pragma is being matched. The
matching process is split into two phases.

\begin{description}
\item [Phase 1]: In Phase 1 the framework tries to match the content of the
pragma against the EBNF specification provided by the user. This is implemented
using a standard backtracking engine which consumes the input stream until the
``end of directive'', {\tt clang::tok::eod}, token is encountered. If the
matching cannot be performed, an error is produced and the pragma is discarded.

If instead the pragma is correctly matched, an instance of the user provided
object type is generated and stored in a list of ``pending'' (or unmatched)
pragmas.

\item [Phase 2]: The Phase 2 takes care of attaching pragmas to the
corresponding statements. Because {\tt LLVM/Clang} processes the pragma before
looking at the target statement (and therefore an AST node is not available by
the time the pragma is processed), the association must be performed lazily.
Because the lack of any context information when a pragma handler is being
invoked, the matching is performed solely on the basis of source code
locations. 
\end{description}


\subsubsection{Pragma Specification}

A pragma specification is provided to the system as an expression built using
C++ operators in a way which resembles the EBNF form. The specification
expression composes a tree which is implemented following the composite design
pattern for which the \type{frontend::pragma::node} class (defined in
\file{frontend/pragma/matcher.h} is the abstract base. The leaves of the
generated tree are lexer tokens. A pragma specification can be built using the
following 4 operators:

\begin{description}
\item [{\tt t1 >> t2}:] Binary operator which represents the concept of
\emph{``concatenation''}, it matches the input stream the next two tokens are
respectively {\tt t1} followed by {\tt t2};

\item [{\tt t1 | t2}:] Binary operator which represents the concept of
\emph{``choice''}, it matches the input stream if the next token is either token
{\tt t1} or token {\tt t2};

\item [{\tt !t}:] Unary operator which represents the concept of
\emph{``option''}, it matches the input stream if the token {\tt t} is present 0
or 1 times;

\item [{\tt *t}:] Unary operator which represents the concept of
\emph{``repetition''}, it matches the input stream if the token {\tt t} is
present 0 or N times;
\end{description}

In each case a token {\tt t} can be either a lexer token (leaf node of the
expression) or an expression. Using these operators is possible to define, for
example, the {\tt for} clause of an OpenMP for (the full code can be found in
\file{frontend/omp/omp\_pragma.cpp}.

\begin{srcCode}
auto kind =  
	Tok<clang::tok::kw_static>() | kwd("dynamic") | kwd("guided") | 
	kwd("auto") | kwd("runtime");

auto op = tok::plus | tok::minus | tok::star | tok::amp |
		  tok::pipe | tok::caret | tok::ampamp | tok::pipepipe;

auto var_list = var >> *(comma >> var);

auto reduction_clause = kwd("reduction") >> 
	tok::l_paren >> op >> tok::colon >> var_list >> tok::r_paren;

auto for_clause =	
	    reduction_clause
	|	(kwd("schedule") >> tok::l_paren >> kind >> 
		!( tok::comma >> expr ) >> tok::r_paren)
	|	(kwd("collapse") >> tok::l_paren >> expr >> tok::r_paren)
	|   kwd("ordered") | kwd("nowait") 
	;
\end{srcCode}

As already stated, the leaf nodes of the expression are lexer tokens which are
imported from {\tt LLVM/Clang} token definition (see in the
\file{clang/Basic/TokenKinds.def}) and made available under the
\decl{frontend::pragma::tok} namespace. Additionally to the preprocessor tokens
of clang we define new leaf nodes for the purpose of simplifying the
specification of new pragmas. 

\begin{description}
\item [kwd( str\_lit ):] the matcher expect to encounter an identifier which is
exactly the literal provided as argument. Note that because we use the C lexer,
keywords which are recognized to be reserved words in the C/C++ language cannot
be matched in this way. In such cases the name of the Clang token must be used,
for example {\tt clang::tok::kw<static>} represent the ``static'' identifier. 

\item [expr:] This placeholder matches any C/C++ expression. Indeed, usually
pragmas may contain expressions. One limitation of the {\tt LLVM/Clang} pragma
matching mechanism is that the C parser is not made available to pragma
handlers. However this is more of a Clang design limitation rather then
capabilities. In order to overcome this problem Insieme works on a patched
version of the {\tt LLVM/Clang} compiler.  The patch makes the engine for pragma
matching of insieme friend (in a C++ way) with the \type{clang::Parser} object.
This enable us to be able to access to the Clang parser also when pragma
handlers are invoked and therefore be able to parse complex C expression without
reinventing the wheel. 

\item [numeric\_constant:] Another useful placeholder which specifies that the
token to match must be any valid numeric constant. 

\item [var:] Makes sure that the matched token is a valid variable. This not
only assures that the token is a valid C identifier, but also that the variable
has been declared. Use of undeclared variable will be captures as an error by
the preprocessor. 

\end{description}


Matching the structure of the pragma is just a part of the whole story. Indeed,
the user may be interested not only to know that a statement has associated a
particular type of pragma, but, most likely, it may be interested to its content. 
Usually, the information contained in the pragma
are mostly syntactic sugar which, once the pragma has been matched, loose any
function. Because the framework cannot decide by itself what is
interesting and not for the user to be stored, we define two additional
operators which allows the user to specify what should be extracted from a
particular pragma once is matched. 

\begin{description}
\item [{\tt["key"]}:] At any point of the pragma specification the {\tt []}
operator can be used to force the framework to store all the tokens matched by
the node to which the operator is applied. Informations are store into a
multimap where the value of the key is the string value provided as argument of
the {\tt []} operator. 
For example, \srcCodeInl{(var >> *(comma >> var))["VARS"]} stores all matched
tokens into a map having {\tt "VARS"} as a key and the list of matched tokens as
value. If the following input is encountered: \srcCodeInl{a, b, c} the resulting
map will be of the form: {\tt ("VARS" -> \{ a, b, c, "," \})}. 

\item [\~{}:] As seen before, sometimes we want to be able to {\em exclude}
specific type of tokens to be mapped to the resulting result. This is the
purpose of the \~{} operator which forces any token mapped by the addressed
expression to be removed from the mapping. Therefore, by changing our
specification for variable lists to: \srcCodeInl{(var >> *(~comma >>
var))["VARS"]}, the resulting multimap will be the following:  {\tt ("VARS" ->
\{a, b, c\})}.

\end{description}

The result of the matcher is therefore an object of type
\type{frontend::pragma::MatchMap}, which is defined in
\file{frontend/pragma/matcher.h}. Each key is matched to a list of objects which
can be either a pointer to a string (used when the pragma matches string
literals for example) or a pointer to a generic \type{clang::Stmt*}. This is the
case when the matched token is a C expression or for example a variable
identifier. It is worth noting that by default keywords nodes are inserted into
the matching map without the need for the user to explicitly specify the
mapping. A \srcCodeInl{kwd("auto")} for example will create an entry in the map
whose key is {\tt "auto"} and the value is an empty list. A recurring use case
is represented in the following code snippet:

\begin{srcCode}
auto var_list = var >> *(~comma >> var);

auto private_clause = 
	kwd("private") >> tok::l_paren >> var_list["private"] >> tok::r_paren;

auto for = 
	kwd("for") >> !private_clause;
\end{srcCode}

Where the {\tt "private"} keyword is utilized to capture the list of variables
associated to the clause. Given the following pragma \srcCodeInl{#pragma omp for
private(a,b)} the generated matching map will be the follow: {\tt "private" -> \{
a, b \}; "for" -> \{ \} }

