\subsection{User Pragmas [Simone]}
\label{sec:Insieme.Frontend.Pragmas}

One of the main features of the Insieme frontend is the ability to easily define
new pragmas. For this purpose a framework has been written which allows the
definition of user pragmas similarly to EBNF form. The framework takes care of
matching pragmas in the input code against the specification given by the user.
If the pragma cannot be matched, an error is produced. Otherwise, if the pragma
is compatible with the EBNF specification, an annotation node is automatically 
generated and associated to the corresponding Clang AST to which the pragma is 
referring. 


\subsubsection{Registering Pragma Handlers}

The implementation of the pragma handling framework is located in the namespace
\decl{frontend::pragma}. The entry point of the framework is the
\type{BasicPragmaHandler<T>} class, define in \file{frontend/pragma/pragma.h},
which is the handler object being registered to the {\tt LLVM/Clang} parser to
be invoked when a pragma is encountered in the input file. In order to
facilitate the creation of such handler objects, the
\type{frontend::pragma::PragmaHandlerFactory} class is defined. 

An example of how a new pragma handler is specified follows:
\begin{srcCode}
clang::PragmaNamespace* omp = new clang::PragmaNamespace("omp");
pp.AddPragmaHandler(omp);

PragmaHandlerFactory::CreatePragmaHandler<OmpPragmaSection>(
	pp.getIdentifierInfo("section"), tok::eod, "omp"
);
\end{srcCode}

First of all {\tt LLVM/Clang} needs a \type{clang::PragmaNamespace} object to be
created which represents the base-name of the pragma. This must be the string
which follows the ``\#pragma'' keyword in the input program. In the example we
define an handler for \srcCodeInl{#pragma omp section}, therefore the namespace
is defined to be \srcCodeInl{omp}. After the namespace is created, we register
it by adding the handler to the current preprocessor ({\tt pp}) which can be
retrieved by the \type{frontend::ClangCompiler} class (note that the Clang
preprocessor shall take ownership of the provided pointer, therefore there is no
need to cleanup the memory, this will be done by Clang once the preprocessor is
destroyed). The code then generates a specific handler for the ``section''
keyword using the \type{PragmaHandlerFactory}. The registration requires the
user to provide the keyword for which this handler must be invoked, an EBNF
specification (which we will explain later), and the name of the namespace
provided as a string. Additionally a template parameter must be provided which
represent the class being instantiated to hold the informations contained on the
matched pragma (\type{OmpPragmaSection} in the example). 

For new pragmas, the user should define a class in order to process and store
the user data contained in the pragma itself. The class
\type{frontend::pragma::Pragma} defined in \file{frontend/pragma/pragma.h}
provide a base class for such purpose. A pragma is defined to store the location
of the start and end location and a reference to the Clang node to which the
pragma was attached. In C a pragma can refer to two kind of nodes, either
declarations (e.g. \type{clang::FuncDecl}, \type{clang::VarDecl}) or a generic
statement. The methods \type{isDecl()} or \type{isStmt()} of the
\type{frontend::pragma::Pragma} class shall be utilized to test the type of the
target node. The methods \type{getDecl()} or \type{getStmt()} can be used to
retrieve a pointer to the target node. 

\subsubsection{Overview of Pragma Matching}

The constructor of the class provided to the \type{CreatePragmaHandler} function
is invoked automatically by the framework once a pragma is being matched. The
matching process is split into two phases.

\begin{description}
\item [Phase 1]: In Phase 1 the framework tries to match the content of the
pragma against the EBNF specification provided by the user. This is implemented
using a standard backtracking engine which consumes the input stream until the
``end of directive'', {\tt clang::tok::eod}, token is encountered. If the
matching cannot be performed, an error is produced and the pragma is discarded.

If instead the pragma is correctly matched, an instance of the user provided
object type is generated and stored in a list of ``pending'' (or unmatched)
pragmas.

\item [Phase 2]: The Phase 2 takes care of attaching pragmas to the
corresponding statements. Because {\tt LLVM/Clang} processes the pragma before
looking at the target statement (and therefore an AST node is not available by
the time the pragma is processed), the association must be performed lazily.
Because the lack of any context information when a pragma handler is being
invoked, the matching is performed solely on the basis of source code
locations. 
\end{description}


\subsubsection{Pragma Specification}

A pragma specification is provided to the framework as an expression built using
C++ operators in a way which resembles the EBNF form. The specification
expression composes a tree which is implemented following the composite design
pattern for which the \type{frontend::pragma::node} class (defined in
\file{frontend/pragma/matcher.h} is the abstract base. The leaves of the
generated tree are lexer tokens. A pragma specification can be built using the
following 4 operators:

\begin{description}
\item [{\tt t1 >> t2}:] Binary operator which represents the concept of
\emph{``concatenation''}, it matches the input stream the next two tokens are
respectively {\tt t1} followed by {\tt t2};

\item [{\tt t1 | t2}:] Binary operator which represents the concept of
\emph{``choice''}, it matches the input stream if the next token is either token
{\tt t1} or token {\tt t2};

\item [{\tt !t}:] Unary operator which represents the concept of
\emph{``option''}, it matches the input stream if the token {\tt t} is present 0
or 1 times;

\item [{\tt *t}:] Unary operator which represents the concept of
\emph{``repetition''}, it matches the input stream if the token {\tt t} is
present 0 or N times;
\end{description}

In each case a token {\tt t} can be either a lexer token (leaf node of the
expression) or an expression tree. Using these operators is possible to define, for
example, the {\tt for} clause of an OpenMP for (the full code can be found in
\file{frontend/omp/omp\_pragma.cpp}.

\begin{srcCode}
auto kind =  
	Tok<clang::tok::kw_static>() | kwd("dynamic") | kwd("guided") | 
	kwd("auto") | kwd("runtime");

auto op = tok::plus | tok::minus | tok::star | tok::amp |
		  tok::pipe | tok::caret | tok::ampamp | tok::pipepipe;

auto var_list = var >> *(comma >> var);

auto reduction_clause = kwd("reduction") >> 
	tok::l_paren >> op >> tok::colon >> var_list >> tok::r_paren;

auto for_clause =	
	    reduction_clause
	|	(kwd("schedule") >> tok::l_paren >> kind >> 
		!( tok::comma >> expr ) >> tok::r_paren)
	|	(kwd("collapse") >> tok::l_paren >> expr >> tok::r_paren)
	|   kwd("ordered") | kwd("nowait") 
	;
\end{srcCode}

As already stated, the leaf nodes of the expression are lexer tokens which are
imported from {\tt LLVM/Clang} token definitions (see in the
\file{clang/Basic/TokenKinds.def}) and made available under the
\decl{frontend::pragma::tok} namespace. Beside to the preprocessor tokens of
{\tt LLVM/Clang} we define several new leaf nodes for the purpose of simplifying
the specification of new pragmas. 

\begin{description}
\item [kwd( "str\_lit" ):] the matcher expect to encounter an identifier which is
exactly the literal provided as argument. Note that because we use the C lexer,
keywords which are recognized to be reserved words in the C/C++ language cannot
be matched in this way. In such cases the name of the {\tt LLVM/Clang} token
must be used, for example keyword {\tt kwd("static")} is not allowed as this is
a reserved keyword in C. The {\tt clang::tok::kw<static>} must instead be used. 

\item [expr:] This placeholder matches any C/C++ expression. Indeed, usually
pragmas may contain expressions. One limitation of the {\tt LLVM/Clang} pragma
matching mechanism is that the C parser is not made available to pragma
handlers. However this is more of a Clang design limitation rather then
capability. In order to overcome this problem Insieme works on a patched version
of the {\tt LLVM/Clang} compiler.  The patch makes the engine for pragma
matching of insieme be able to retrieve the instance of the parser
(\type{clang::Parser)} object.  This allows Insieme to invoke the parser even
during the processing of pragmas (which is not allowed by relying solely on {\tt
LLVM/Clang} API. This means that complex C/C++ expressions can be used in the
pragma specifications and semantics check can be automatically performed on
those (e.g. use of undeclared variable). 

\item [numeric\_constant:] Another useful placeholder which specifies that the
token to match must be any valid numeric constant. 

\item [var:] Makes sure that the matched token is a valid variable. This not
only assures that the token is a valid C identifier, but also that the variable
has been declared. Use of undeclared variable will be captures as an error by
the preprocessor. 
\end{description}

\subsubsection{Pragma Matching}

The matcher uses the expression and invoke the \type{match()} virtual function
which has a different implementation for every type of connector. In this way we
are able to implement a sort of backtracking engine which tries to consume the
input stream until a match is obtained. If not, an error message is
automatically produced listing the alternatives which the matcher was expecting
at the specific location and it couldn't find instead. The error message is
returned back to the user using the {\tt LLVM/Clang} diagnostic engine showing
the precise code location where the error occurred. 

Matching the structure of the pragma is just a part of the whole story. Indeed,
the user may be interested not only to know that a statement has associated a
particular type of pragma, but, most likely, it may be interested to its
content.  Usually, the information contained in the pragma are mostly syntactic
sugar which, once the pragma has been matched, loose any function. Because the
framework cannot decide by itself what is interesting and not for the user to be
stored, we define two additional operators which allows the user to specify what
should be extracted from a particular pragma once is matched. 

\begin{description}
\item [{\tt["key"]}:] At any point of the pragma specification the {\tt []}
operator can be used to force the framework to store all the tokens matched by
the node to which the operator is applied. Informations are store into a
multimap where the value of the key is the string value provided as argument of
the {\tt []} operator. 
For example, \srcCodeInl{(var >> *(comma >> var))["VARS"]} stores all matched
tokens into a map having {\tt "VARS"} as a key and the list of matched tokens as
value. If the following input is encountered: \srcCodeInl{a, b, c} the resulting
map will be of the form: {\tt ("VARS" -> \{ a, b, c, "," \})}. 

\item [\~{}:] As seen before, sometimes we want to be able to {\em exclude}
specific type of tokens to be mapped to the resulting result. This is the
purpose of the \~{} operator which forces any token mapped by the addressed
expression to be removed from the mapping. Therefore, by changing our
specification for variable lists to: \srcCodeInl{(var >> *(~comma >>
var))["VARS"]}, the resulting multimap will be the following:  {\tt ("VARS" ->
\{a, b, c\})}.

\end{description}

The result of the matcher is therefore an object of type
\type{frontend::pragma::MatchMap}, which is defined in
\file{frontend/pragma/matcher.h}. Each key is matched to a list of objects which
can be either a pointer to a string (used when the pragma matches string
literals for example) or a pointer to a generic \type{clang::Stmt*}. This is the
case when the matched token is a C expression or for example a variable
identifier. It is worth noting that, by default, keywords nodes are inserted into
the matching map without the need for the user to explicitly specify the
mapping. A \srcCodeInl{kwd("auto")} for example will create an entry in the map
whose key is {\tt "auto"} and the value is an empty list. A recurring use case
is represented in the following code snippet:

\begin{srcCode}
auto var_list = var >> *(~comma >> var);

auto private_clause = 
	kwd("private") >> tok::l_paren >> var_list["private"] >> tok::r_paren;

auto for = 
	kwd("for") >> !private_clause;
\end{srcCode}

Where the {\tt "private"} keyword is utilized to capture the list of variables
associated to the clause. Given the following pragma \srcCodeInl{#pragma omp for
private(a,b)} the generated matching map will be the follow: {\tt "private" -> \{
a, b \}; "for" -> \{ \} }

The generated map is forwarded to the object constructor registered through the
{\tt PragmaHandlerFactory}. The pragma object is constructed iff the matcher is
able to completely match the user specification against the input code. The user
object is responsible to analyze the matching map returned by the pragma matcher
and do the necessary operations. An explanatory example of how OpenMP pragmas
are processed and stored is in the \file{frontend/pragma/omp/omp\_pragma.cpp}
file. 

\subsubsection{AST Node Mapper}

The second phase of the pragma framework takes care of associating, or mapping,
generated pragma objects to the AST nodes to which pragmas refer. The framework
does this operation in a way which is transparent to the user. The code which
takes care of this aspect is in the \file{frontend/sema.h} and
\file{frontend/sema.cpp} files. 

Two solutions are possible in order to connect pragmas with statements; one
solution would be to traverse the entire AST after it has been generated.
However this solution requires an expensive traverse of the input program AST
and this could be inefficient for big codes, especially since the
pragma/statement ratio is usually very small. 

The way the {\tt LLVM/Clang} compiler builds the AST is by invoking ``actions''
provided by the \type{clang::Sema} class which reduces the tokens currently
available on the parser stack and generates the corresponding AST node, and
append it to the program AST.
This is the best location for implementing our matching algorithm for pragmas.
Indeed we can keep the list of processed pragmas and every time a new
statement is being created by the \type{clang::Sema} class we check, based on
the location, whether any of the pending pragmas refer to the newly generated
statement. With this solution the overhead produced by the matching is minimal
as our lookup procedure is local to the code section defining pragmas. For code
segments containing no pragma we do not pay any overhead. 

Fortunately, when the first version of insieme was developed, {\tt LLVM/Clang}
interfaces allowed for the class \type{clang::Sema} to be extended. As a matter
of fact all the action methods were virtual allowing for extending its
behaviour.  Starting with {\tt LLVM/Clang} 2.9 the interfaces had a dramatical
change the {\tt LLVM/Clang} developers removed the possibility to customize the
actions of the \type{clang::Sema} class. In order to overcome this limitation
imposed by the {\tt LLVM/Clang} developer we patch the clang code making virtual
the functions of the \type{clang::Sema} class for which we need to redefine the
behaviour. 

As stated before, the matching algorithm works with the only context information
available at this stage, i.e. locations. The check for pragma matching is
performed for few selected node types, i.e. {\tt Act\-On\-Compound\-Stmt}, {\tt
Act\-On\-If\-Stmt},{\tt Act\-On\-For\-Stmt}, {\tt
Act\-On\-Start\-Of\-FunctionDef}, {\tt Act\-On\-Finish\-Function\-Body}, {\tt
Act\-On\-Declarator}, {\tt Act\-On\-Tag\-Finish\-Definition}.  The algorithm
keeps a list of pending pragmas ordered by their locations. Once one of the
statements is reduced, we check which pragmas are within the range of
the statement. If none, we check whether any of the pending pragmas are located
right before the statement. In that case the pragma gets associated to that
statement and we remove the pragma from the list of pending pragmas. If pragmas
are located within the range of the statement we iterate through the children
and match accordingly to the positions.

\subsubsection{Detached Pragmas} 
One tricky aspect of the entire algorithm is how we deal with pragmas which are
not meant to be attached to a statement. An example is the OpenMP \srcCodeInl{#pragma
omp barrier}. Indeed this pragma is not meant to be associated to a statement,
for example the following code is a valid OpenMP input code:

\begin{srcCode}
{
	...
	#pragma omp barrier
}
\end{srcCode}

Because the {\tt LLVM/Clang} compiler doesn't represent pragmas in the AST we
need a way to easily map the location of a pragma to a node in the AST. So that
when the AST is traversed for the IR conversion, we can handle the pragma.
However, when a pragma does not not refer to a statement the matching algorithm
creates an empty statement (a {\tt no-op}, {\tt ;}) and transform the {\tt
LLVM/Clang} AST by inserting the no-op at the correct location. We then map the
pending pragma to the generated statement. In order to check whether
the statement associated to a pragma is generated by Insieme it is necessary to
query the statement for its location. If the returned
\type{clang::SourceLocation} object is not valid, then it means this statements
has been introduced by the matching algorithm. 

\subsubsection{Traverse (and Filter) Pragmas}

As depicted in Figure~\ref{fig:Frontend.Translation.Units}, each
\type{frontend::TranslationUnit} object has a reference to a list of Pragmas
object being generated by the clang frontend and the Insieme pragma handling
framework. Therefore given a translation unit, we can retrieve the list of
pragmas in that unit. An example is the following: 

\begin{srcCode}
NodeManager manager;
frontend::Program prog(manager);
frontend::TranslationUnit& tu = prog.addTranslationUnit( "input.c" );
for(const PragmaPtr& cur : tu.getPragmaList()) { LOG(INFO) << *cur; }
\end{srcCode}

However, it is usually more useful to retrieve the complete list of pragmas
across translation units. This can be done through the \type{frontend::Program}
class which offers two methods \decl{pragmas\_begin()} and \decl{pragmas\_end()}
which returns an iterator through all pragmas in the input program. Additionally
a filter can be passed so that the user obtains only pragmas of a certain
category. For example iterating through the all {\tt "insieme::mark"} pragmas
can be easily done as follows:

\begin{srcCode}
auto pragmaMarkFilter = [](const pragma::Pragma& curr) { 
	return curr.getType() == "insieme::mark"; 
};
for(Program::PragmaIterator pIt = pragmas_begin(pragmaMarkFilter), 
							pEnd = pragmas_end(); pIt != pEnd; ++pIt) 
{ ... }
\end{srcCode}

