\subsection{Command Line Arguments [Simone]}

\label{Command.Line.Args}

A compiler usually comes with a number of input options which can be used to
drive the behaviour of the compiler. For example, by enabling the user to
trigger different analysis or optimizations during the compilation process.
In Insieme input arguments are managed by the {\tt Boost.Program\_options}
library \cite{boost_program_options}. However, in order to make the operation of
adding new input arguments/flags to the compiler easy, the library has been
wrapped into a framework which allows the specification of input flags via a
simple XMacro files. 

The file \file{utils/cmd\_line\_utils.h} contains the definition of the
\type{utils::CommandLineOptions} class. This class is populated by the C preprocessor
using the content of the XMacro file \file{utils/options.def}. Each line of the
file contains the definition of an input argument of the Insieme compiler. The
file is divided into 3 sections containing respectively: 

\begin{description}
\item [Flags]: A flag can appear or not appear in the command line. When it
appears its value is set to the boolean value true, otherwise its value is
false. An example is the {\tt --help or -h} or the {\tt --check-sema or -S}
flags. 

\item [Integer Options]: This is an optional argument which can assume an
integer value. When the option does not appear in the input argument it is set
to a default value. An example is the option used to set the verbosity level of
the compiler, i.e. {\tt -v 1} or {\tt --verbose=1}. 

\item [Options]: Very similar to the Integer counter-part but the value of an
option can be either a string or a vector of strings.
\end{description}

An example of the structure of the \file{utils/options.def} XMacro file is the
following:

\begin{insCode} 
/* FLAG(opt_name, var_name, def_value, var_help) */
FLAG("help,h",    Help,     false,     "produce help message")
...
/* INT_OPTION(opt_name, var_name, def_value, var_help) */
INT_OPTION("verbose,v", Verbosity, 0, "Set verbosity level")
...
/* OPTION(opt_name, var_name, var_type, var_help) */ 
OPTION("out-file,o", Output,  std::string, "output file")

OPTION("include-path,I", IncludePaths, std::vector<std::string>, 	
	   "Add directory to include search path")
...
\end{insCode}

An input flag for the compiler is defined by the {\tt FLAG()} macro. This macro
accepts 4 arguments, all of them are mandatory and none is optional. 
\begin{itemize}
\item The first argument is the name of the flag. Two versions are provided:
full and abbreviated. For example the definition {\tt "help,h"} defines two
input arguments in the form {\tt -h} or {\tt --help} both connected to the same
input flag.  

\item The second argument is the variable name which will be
utilized to store the value of the flag. The name of this variable MUST be
unique. 

\item The third argument is the default value of the which the flag should
assume when not specified in the command line. Usually you want to set this
value to false, however some flags can be forced to be set to true by default. 

\item The last argument is a string literal containing the message which explain
the semantics of the defined flag. This message is shown whenever the {\tt
--help} option is invoked on the Insieme compiler executable. 
\end{itemize}

Definition of an integer option is similar to the one for flags. The only
difference is that the default value can be any integer value. For generic
options we do not provide, for now, the ability of specifying the default value
but instead the third argument is utilized to specify the type associated to
that option.  Valid types for options can be \type{std::string},
\type{std::vector<std::string>}, \type{int}, \type{std::vector<int>}. List
options capture the use case where a particular input argument can be repeated
multiple times. An example is the include path ({\tt -I}, of the list of
preprocessor definitions {\tt -D}.  Whereas string options must be used when a
single instance is allowed in the argument list. 

Once a flag or option is defined, the framework generates automatically a number
of {\tt static} fields in the \type{utils::CommandLineOptions} class. Exactly
one field per option/flag is provided and the name of the field is the value of
the second argument specified in the XMacro file (that's the reason why names of
command options must be unique). Population of those fields is taken care by the
{\tt utils::Parse(int argc, char* argv[])} method also defined in
\file{utils/cmd\_line\_utils.h}. After the \decl{Parse()} function is invoked,
the content of the static fields of the \type{utils::CommandLineOptions} class
can be accessed to retrieve the value of a specific flag or option. The decision
of saving those information as static fields has been motivated by the fact that
command line arguments should be available and easy to access at any phase of
the compilation process. Therefore, in order to avoid to pass the
\type{utils::CommandLineOptions} object everywhere within the compiler, the
singleton solution was preferred (by most). Having a singleton at this stage may
however interfere with parallelization, therefore the developer must take care
of maintaining consistency. 

\subsubsection{How to Add a New Command Line Option/Flag?}
Add the new definition in the \file{utils/options.def} file specifying all the
required informations (as shown in the previous example).

\subsubsection{How to Access to Command Line Option/Flag's Value?}
The \decl{Parse()} method is invoked in the Insieme main function before
anything is performed. Therefore the value of the input arguments are
immediately available to all the modules of the compiler. An example of how to
access command line arguments is well documented in the main function of the
Insieme compiler in \file{driver/main.cpp}. However an example is the following:

\begin{srcCode}
// Parse input arguments and populate the CommandLineOptions object
utils::Parse(4, { "./a.out", "-I.", "-I/usr/include", "-v=2"}); 

for(const std::string& path : utils::CommandLineOptions::IncludeFiles) {
	std::cout << path << std::endl;
}
// Prints to standard output: ".", "/usr/include"

if (utils::CommandLineOptions::Verbosity>2) {
	std::cout << "Very verbose output" << std::endl;
}
\end{srcCode}

\subsubsection{Traps and ``Non-Standard'' Use Cases}
A known trap is related to availability of input arguments when the compiler is
instantiated within the compiler like in test cases. Because there are no input
arguments to parse, \type{CommandLineOptions}' field will not be correctly
initialized and compiler code which depends on the value of those fields may
stop working. 

There are two ways of correctly dealing with this problem. You can manually
build the arguments for the \decl{Parse(...)} function as shown above.  The
second way is to set the value of \type{CommandLineOptions}' field manually
before invoking the desired compiler module. An example of this solution to the
problem was implemented to provide a \emph{facade} interface to the frontend
(\file{frontend/frontend.h}).
