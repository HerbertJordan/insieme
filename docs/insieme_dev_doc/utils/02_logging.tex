\subsection{Logging [Simone]}
\label{sec:Insieme.Logging}

Logging utilities for Insieme are defined in the \file{utils/logging.h} header
file. The logger provides basic functionalities for logging at different levels
of priority (i.e. {\tt DEBUG, WARNING, INFO, ERROR, FATAL} and verbosity (i.e.
$1-N$). The main characteristic of the logger is that it is \emph{thread safe}. This
means that when multiple threads compete for the output stream, the logger 
makes sure that the output is not overlapping.  

Compared to other logging utilities, which allows to load the configuration from
a file, or save the output in multiple formats, our logger is much simple but at
the same time it is designed to have very low overhead and it is memory leak
free. 

\subsubsection{Instantiate the Logger}
Because the logger should be available from any program point and only one
instance of the logger is necessary during the lifetime of the compiler (unless
future user scenarios arise), we use the singleton design pattern. The logger is
initialized at the entry point of the program using the
\srcCodeInl{Logger::get(std::ostream\& out, const Level\& level=INFO, unsigned
short verbosity=CommandLineOptions::Verbosity)} method. Successive calls to
this method will return the singleton object connected to the logger. 

\begin{srcCode}
#include "insieme/utils/logging.h"
log::Logger::get(std::cerr, log::INFO);
\end{srcCode}

The {\tt level} and {\tt verbosity} parameters define the priority level and the
verbosity at which the logger works. This means that a logger instantiated with
the priority level {\tt WARNING}, for example, will only print to the output stream
logging messages at level {\tt WARNING} or higher. The verbosity level is
meaningful only when the priority level is set to {\tt DEBUG}. For higher levels
verbose messages are indeed always skipped. 

\subsubsection{Using the Logger}
The way the logger is meant to be is via the set of utility macros defined at
the bottom of the \file{utils/logging.h} file. The {\tt LOG} and {\tt VLOG}
macros should be used for respectively logs a message with a specific level of
priority level and debug logs with varying verbosity level. An example of how
the logging macros can be used follows:

\begin{srcCode}
#include "insieme/logging.h"

LOG(INFO) << "This is an important message!";
LOG(DEBUG) << "Debug log is activated";

VLOG(1) << "Show this message when the verbosity level is >= 1";
\end{srcCode}

The output of this code depends on the initial settings of the logger. If the
logger is initialized at level {\tt ERROR} for example, none of the messages
will be printed. However, if the log is initialized in {\tt DEBUG} mode and with
a verbosity level greater than 1 all three messages will be produced to the
logger stream. 

It is worth noting that the logger always append a new-line at the end of a
message. Additionally, for performance reason, the output of the macro is not a
reference to an output stream therefore is not possible to use the {\tt LOG} and
{\tt VLOG} macros as input arguments for functions. However for situation where
the normal behaviour of the logger needs to be overwritten, the 
{\tt LOG\_STREAM(LEVEL)} macro return a C++ object which implements the
\type{std::ostream} interface. For example this object allow to eliminate the
default new line at the end of a logging message, e.g.:

\begin{srcCode}
#include "insieme/logging.h"
LOG_STREAM(INFO) << "Starting fancy process...";
do_something_fancy();
LOG_STREAM(INGO) << "Done!" << std::endl;
\end{srcCode}

Additionally, it is possible to use the object returned by the {\tt LOG\_STREAM}
macro as input to functions accepting a generic output stream. 

\begin{srcCode}
#include "insieme/logging.h"

std::vector<int> vec { 10, 20, 30 };
std::copy(vec.begin(), vec.end(), 
	std::ostream_iterator<int>( LOG_STREAM(DEBUG), "-" )
);
\end{srcCode}

However {\bf DO NOT OVERUSE} the {\tt LOG\_STREAM} macro as it might penalize
performance. Indeed, while the {\tt LOG} macro is written in a way that the
compiler can remove the logging line because it detects to be dead code.  When
the {\tt LOG\_STEAM} macro it utilized, an handler is returned which simply dump
all the messages. Therefore printing a message at a lower priority still costs
execution time (even if nothing gets printed to the output stream). 

\subsubsection{How to Customize the Logger Line Preamble?}

The Logger also automatically adds a preamble to each line containing
informations which may be useful to debug the code. The structure of the
preamble is configurable (as almost everything else in Insieme). 

\begin{srcCode}
INFO  main.cxx:484] Insieme compiler
\end{srcCode}

This is done by the \type{utils::log::Formatter} class which statically builds
specifier for the logger preamble. A preamble can be built by nesting
specifiers, at each level a specifier requires a separator char and a list of
elements which composes the specifier. The default specifier is defined as
follows:

\begin{srcCode}
Formatter<' ',LevelSpec<LEVEL>,
	Formatter<':',FileNameSpec<0>,LineSpec<__LINE__>>
>
\end{srcCode}

This formatter specify that the preamble is composed by the current log level
and a second formatter which contains the file name and the current line. The
outermost formatter uses a space as separator while the innermost uses a colon
symbol. The formatter is instantiated every time the {\tt LOG} macro is
executed. If the user needs to change the look of the logger preamble this can
be easily done by either overriding the {\tt MAKE\_FORMAT} macro or by by
defining an overloaded version of the {\tt LOG} macro. For example, it is
possible to include time information in the logger preamble by defining the
following formatter:

\begin{srcCode}
#undef MAKE_FORMAT

#define MAKE_FORMAT(LEVEL) \
Formatter<' ',LevelSpec<LEVEL>, TimeSpec<FULL>, \
	Formatter<':',FileNameSpec<0>,LineSpec<__LINE__>> \
>
\end{srcCode}

After this redefinition, the output of the {\tt LOG} method will look different:
\begin{srcCode}
INFO  2012-Jun-15 18:17:35 main.cxx:484] Insieme compiler
\end{srcCode}

This feature can be used to define loggers which include, in the case of
multi-threaded code, the thread ID in order to simplify debugging of the
compiler code. 


\subsubsection{Support for Multi-Threading}

One of the main feature of the logger implemented in Insieme is the fact that it
is thread-safe. As a matter of fact the logger uses an internal mutex to make
sure threads writing to the logger do not overlap and thus generate unreadable
debugging messages. 

The main idea is on the use of C++ temporary object properties to maintain
exclusive access to the logger stream by competing threads. Every time the {\tt
LOG} macro is invoked a temporary object is instantiated which retain the lock
to the mutex guarding the logger stream. As long the object is alive, the lock
is retained and no other thread can use the output stream. The lock is released
when the object is being destroyed thereby allowing other thread to access the
stream. 

Invocation of the {\tt LOG\_STREAM} macro returns a temporary object which
retains a lock to the stream. It is needles to say that playing with {\tt
LOG\_STREAM} may lead to deadlock situation. Therefore be use the {\tt
LOG\_STREAM} macro carefully and only when strictly necessary. The logger uses a
recursive mutex which allow the same thread to acquire lock to the same mutex
several time therefore avoiding deadlock in those situations. 

For any other implementation details of the Insieme logger, the code is your
friend, just play with it.
 
